\documentclass[a4paper, 12pt]{article}
\usepackage[top=2cm, bottom=2cm, left=2.5cm, right=2.5cm]{geometry}
\usepackage[utf8]{inputenc}
\begin{document}

\begin{center}
 \textbf{Equação polinomial do 2ºgrau}
\end{center}
 
\begin{flushright}
 \textit{Equação polinomial do 2ºgrau}
\end{flushright}
 
\begin{flushleft}
 \underline{Equação polinomial do 2ºgrau}
\end{flushleft}

\begin{flushleft}
 \textbf{\textit{\underline{Equação polinomial do 2ºgrau}}}
\end{flushleft} 

 Uma equação da forma $$ax^2 + bx + c = 0,$$ com $a \neq 0$ será chamada de equação polinomial do 2º grau
 A solução dessa equação é dada por 
 $$ = \frac{-b \pm \sqrt{b^2 - 4ac}}{2a}$$ 
\end{document}